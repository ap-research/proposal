Consider using the following structure for your research proposal. Change the title at the top of this document to reflect your research topic and hypothesis. Replace the subsequent line (AP Research) with your name. You may use this document as a template, but make sure to organize your ideas into paragraph form. The bullet points below are only guiding questions to help you formulate your ideas. Please limit your proposal (including references) to two to four pages single spaced.

\section{Research Topic}

Provide a brief overview of your research topic and question, which will be elaborated in the Hypothesis section. Include the aims and objectives of your proposed research. 

\section{Literature Review}

Your literature review should not just be a simple summary of the previous work related to your research topic. You should synthesize a theoretical framework with associated findings based on previous research in the field. The theoretical framework should help you develop your hypothesis in the next section. Similarly, the associated findings can serve as a starting point for the data and methods you will use to test your hypothesis. You may decide to organize your literature review in one of the following approaches:

\begin{itemize}

\item Organize your sources thematically. Brainstorm a conceptual map of all the theories related to your research topic. Then describe and assess related findings that fall under each theory.

\item Trace the historical development of research related to your topic. How has the theoretical framework evolved in your chosen research topic as a result of different findings from researchers?

\end{itemize}

%natbib citation commands
In-text citation example: \citet*{Aqda11} conducted a study showing that computer-aided instruction boosted math creativity among junior high students.~\citeauthor{Aqda11} examined the pre- and post-test differences in both the control and experimental groups, which received traditional and computer-aid math instruction, respectively.\footnote{In the APA style, sources with three to five authors should be cited with all names in the first reference. In subsequent repeated citations, you should list only the first author followed by the "et al." abbreviation. For sources with six or more authors, use this abbreviated citation starting from the first reference. Also, remove the year in subsequent in-text citations as shown in this example. However, for parenthetical citations, keep the year in all repeated references. These guidelines may vary by citation style.}

Parenthetical citation example: Other studies have suggested that computer-aided instruction does not necessarily lead to higher student achievement~\citep{Angrist02, Ross99}.

At the end of this section, you should identify gaps or deficiencies in the literature as a way to motivate your research question. What is currently missing in the literature that you plan on addressing with your research?


\section{Hypotheses}

In the context of your literature review, you should develop a hypothesis or set of hypotheses that addresses your research question. Your hypothesis will eventually develop into your thesis statement in your research paper. However, at this point, your hypothesis is simply your first attempt to delineate possible explanations for your research question based on existing theories and results from your literature review.


\section{Data}

Identify and describe the data that you plan to use to test your hypothesis. The data could be quantitative, qualitative, or both. At this stage of your research, you should not dig too deeply into the data, as this could bias your analysis by forcing the data to fit your hypothesis. Instead, you should explore a wide range of data sources that will be pertinent to your research question. 

If you plan on using existing data, please cite the data sources. Example: I plan to use the 2010--2012 National Survey of Early Care and Education~\citep{Center2019} to analyze the relationship between early-childhood program participation and subsequent academic achievement.

The following questions and prompts may help you describe the data. If you are planning to collect your own data, you can still use these guiding questions to describe your proposed data-gathering procedures and process.

\begin{itemize}
\item Describe the general classification and characteristics of the dataset.
    \begin{itemize}
    \item Classification examples: survey data, experimental data, observational data, administrative data, census data, historical documents
    \item Characteristic examples: longitudinal testing data of U.S. high school students, cross-sectional data of household socioeconomic indicators aggregated at the county level
    \end{itemize}

\item How was the data collected?
    \begin{itemize}
    \item For survey data, describe the sampling method (e.g., simple random sampling, stratified sampling, cluster sampling).
    \item For administrative data, describe the inclusion criteria (i.e., factors or conditions that led to the collection of the data; e.g., de-identified library usage data captured by self-service kiosks and published on the city's open data web page).
    \item For experimental data, describe the experimental research design.
    \item For qualitative data, describe the collection process such as interviews or fieldwork observations.
    \end{itemize}

\item Describe the results from previous papers that analyzed this dataset or similar data. Your literature review may have included these sources, but you may decide to use this section to elaborate more on the details of the data analysis from previous research. This will help you segue into the following question as well as the Methods section.

\item How will the data help you investigate your research question?

\end{itemize}

\section{Methods}

Briefly describe the methods you will employ to analyze your data. You should provide enough information to demonstrate that your research blueprint is feasible for completion within a one-year timeline.

How will you apply your proposed methods to the data to test your hypothesis? What are the underlying assumptions of the research methods? Do your data meet the criteria of the assumptions for applying these methods? If not, what proper adjustments will you need to make?

Generally, you will not need to cite the source of commonly employed methods, such as basic statistical tests covered in AP Statistics. However, for uncommonly used or more specialized methods, you should cite the sources that originated the methods. If you are emulating or modifying specific methods used in previous research, you need to provide the proper citations to acknowledge the methodological foundations from which you are designing your research.

Most importantly, use this section to address any ethical issues with data collection and research methodology. Researchers need to abide by data privacy standards such as de-identification of individual records. For non-public data, the method of obtaining and storing the data should follow any applicable policies and regulations.

At the end of this section, describe the skills that are needed to conduct this research. If the methods require skills that you currently do not possess, describe the resources you will use to acquire the necessary skills. While your teachers will likely cover general research methods in the AP Research course, you will need to take ownership of your own learning to demonstrate competency as an independent researcher. For more specialized techniques, your teachers may decide to provide individualized advising or conduct mini-workshops for small groups of students conducting similar research. In cases involving highly specialized research, your teachers may work with you to identify another teacher with the subject-area expertise necessary to provide supplementary advising.  

 

